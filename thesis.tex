
%----------------------------------------------------------------------------------
%----Präambel/Preamble-------------------------------------------------------------
%----------------------------------------------------------------------------------

\documentclass[	a4paper,
				11pt,
				DIV=11,
				bigheadings,
				idxtotoc,
				listof=totoc,	
				bibtotoc,		
				halfparskip,
				cleardoubleempty,
				oneside,
				openright]{scrartcl}
%----------------------------------------------------------------------------------

\usepackage[english]{babel}
\usepackage[T1]{fontenc}
\usepackage[utf8]{inputenc}


\usepackage{graphicx}	

\usepackage[labelfont=bf]{caption}
					
\usepackage{float}
\usepackage{wrapfig}
%\usepackage{subfigure}

\usepackage{geometry}						% Für newgeometry in Titelseite
\geometry{a4paper,left=30mm,right=20mm}

\usepackage{blindtext}
\usepackage{layout}

\PassOptionsToPackage{hyphens}{url}		
\usepackage[pdfborder={0 0 0},
			colorlinks=true, 
			linkcolor=black,
			citecolor=red,
			]{hyperref}
			
\usepackage{natbib}%numbers 				

\usepackage{pdfpages}

\usepackage{color}
\usepackage{xcolor}

\usepackage{setspace}

\usepackage{longtable}
\usepackage{multirow}
\usepackage{colortbl}

%----------------------------------------------------------------------------------
%----Kopfzeile---------------------------------------------------------------------
%----------------------------------------------------------------------------------

\usepackage{scrlayer-scrpage} 							% Aufruf KOMA-Skript für Kopfzeilen

\pagestyle{scrheadings}							% Definition der Eigenen Headerformatierung
\clearscrheadfoot 								% alle Standard-Werte und Formatierungen raus
%\automark[chapter]{section}						% Kapitel und Section als Inhalt der Variablen leftmark und rightmark
\ohead{\pagemark}								% Seitenzahl auf äußerem Rand
\ihead{\ifthispageodd{\leftmark}{\rightmark}} 	% Innere Überschrift mit Kapitel bei linker Seite und Section bei rechter Seite -> geht nur in Verbindung mit
												% zweiseitigem Text wirklich sinnvoll
\setheadsepline{0.4pt} 							% Trennlinie Fußzeile und Textkörper
\setkomafont{pagehead}{\scshape}				% Schriftart in Kopfzeile, \scshape = Kapitelchen
%----Fußzeile----------------------------------------------------------------------
\setfootsepline{0.4pt} 							% Trennlinie Fußzeile und Textkörper
\setkomafont{pagefoot}{\scshape}				% Schriftart in Fußfzeile, \scshape = Kapitelchen
\ifoot{\footnotesize{John Doe}}
\ofoot{\footnotesize{Master thesis}}		
%----------------------------------------------------------------------------------

\defpagestyle{myPageStyle}{
	(0pt ,0pt)
	{\hfill\pagemark} {\hfill\pagemark} {\hfill\pagemark}
	(0pt ,0pt)	
}{
{
	(\textwidth ,0.4pt)
	\footnotesize{Eric Horacsek} \hfill \footnotesize{Bachelor thesis}} 
	{\footnotesize{Eric Horacsek} \hfill \footnotesize{Bachelor thesis}} 
	{\footnotesize{Eric Horacsek} \hfill \footnotesize{Bachelor thesis}}
	(0pt ,0pt)
}


%----Farbdefinition--THI-blau------------------------------------------------------
\definecolor{haw_mag}{rgb}{0,0.112,0.47}
\addtokomafont{section}{\color{haw_mag} \rmfamily \scshape} 
\addtokomafont{subsection}{\color{haw_mag} \rmfamily}
\addtokomafont{subsubsection}{\color{haw_mag} \rmfamily}
\addtokomafont{paragraph}{\color{haw_mag} \rmfamily}
\addtokomafont{subparagraph}{\rmfamily}
%----------------------------------------------------------------------------------

\definecolor{tab_2}{RGB}{230,230,230}
\definecolor{tab_1}{RGB}{85,128,214}

%------Längenanpassung-------------------------------------------------------------
\setlength{\headsep}{10mm}						% Textabstand zur Kopfzeile
\setlength{\footskip}{15mm}						% Abstand zur Fußzeile
\setlength{\textheight}{235mm}					% Texthöhe
%----------------------------------------------------------------------------------


%----Glossar-----------------------------------------------------------------------
\usepackage[toc, acronym]{glossaries} 			
\makeglossaries	

%----------------------------------------------------------------------------------
%----Glossar-----------------------------------------------------------------------
\usepackage[intoc]{nomencl} 
\makenomenclature
%----------------------------------------------------------------------------------



\includeonly{	
	titlepage,							
	affidavit,
	acknowledgments,
	%abstractDE,
	abstractEN,
	confidentialityClause,
	glossary,
	mainpart,
	%outlook,
	%fazit,
	nomenclature,
	appendices
	}
	
			
%-------------------------------------------------------------------------------------------------------------------------------------------------------------
%----------------DOKUMENT-BEGINN---------------------------------------------------
%-------------------------------------------------------------------------------------------------------------------------------------------------------------

\begin{document}

	%\shorthandoff{"}						% Vermeidung von ungewollten Ligaturen/Avoid unwanted ligatures
	
	%----Vermeidung von Hurenkindern und Schusterjungen---------------------
	\widowpenalty=10000
	\clubpenalty=10000
	\displaywidowpenalty=10000	
	%-----------------------------------------------------------------------

	%Titelseite/title page	
	%----------Titelseite-------------------------------------------------------------

\newgeometry{textheight=0.9\paperheight, textwidth=0.76\paperwidth, left=30mm, right=20mm}

\begin{titlepage}	
	%----THI+(x-company)-logo--------------------------------------------------------
		\begin{figure}[!h]
			\centering
			\includegraphics[width={0.99\textwidth}]{images/thiRGB.jpg}	
		\end{figure}																			
	%------------------------------------------------------------------------------
	
	\begin{center}
		\hrulefill 
	\end{center}
	
	
	\begin{center}	
		\vspace{1cm}
		
		\huge\textbf{
			Bachelor Thesis}\\[2.5em]
		\normalsize
			Fakultät Informatik	\\ [7em]
	
		\Large\textbf{your thesis tile goes here}	 \\ 

	\end{center}

	\vfill
	
	
	\begin{tabular}{lll}
		First- und Last name &: & \textbf{Eric Horacsek}	\\ [3em]
		
		Registered on &:	& xx.yy.zzzz	\\ [1em] % issuing date
		Submitted on &:	& xx.yy.zzzz	\\ [3em] %date of hand in
		
		First Examiner &: 	& Prof. Dr.-Ing. Georg Passig	\\ [1em]
		Second Examiner &: 	& Prof. Dr. Erika Mustermann	\\[3em]
		
		Company advisor &:	& Mr. The Expert \\ %if applicable
	\end{tabular}
	
\end{titlepage}

\restoregeometry					% include erzeugt immer eine neue Seite bei jedem Einbinden
	\cleardoublepage						% include always creates a new page
	
	\pagenumbering{Roman} 			% Römische Nummerierung der Kapitel/roman page numbering
	
	%Erklärung
	\thispagestyle{myPageStyle}
	\include{affidavit}
	\cleardoublepage
	
	%Danksagung
	\thispagestyle{myPageStyle}
	\include{acknowledgments}
	\cleardoublepage
	
	%Kurfassung/Abstract German (only for thesis written in German)
	%\thispagestyle{myPageStyle}
	%\include{abstractDE}
	%\cleardoublepage
	
	%Kurzfassung/Abstract Englisch (for every thesis)
	\thispagestyle{myPageStyle}
	%----------Zusammenfassung Englisch/Abstract----------------------------------------------------------------
\addsec{Abstrakt}
\onehalfspacing
 {\Large Diese Arbeit beschreibt die Konzeption sowie Realisierung einer mobilen Anwendung zur Echtzeit-Erfassung
 	und -Überwachung von EKG Signalen, welche mittels ein drahtlosen Schnittstelle bereitgestellt werden. Der Schwerpunkt dieser Arbeit liegt auf der implementierung einer robusten Lösung, welche es ermöglicht EKG-Signale kontinuirlich aufzuzeichnen und über eine Bluetooth Low Energy Schnittstelle in Echzeit zu
 	übertragen. Darüber hinaus soll eine Langzeitaufnahme von bis zu 24 Stunden, interpretiert und dargestellt werden, um diese anschließend für spätere Analysen zugägnlich zu machen. Neben der technischen Umsetzung werden Herausforderungen bei der Echtzeit-Datenübertragung und Darstellung sowie der Speicherverwaltung diskutiert. \par }


	\cleardoublepage
	
	%Sperrvermerk/Confidentiality clause (if any)
	\thispagestyle{myPageStyle}
	%----------Sperrvermerk/Confidentiality clause------------------------------------------------------------


\addsec{Confidentiality clause} % Sperrvermerk/

Optional.\\ 
	
Ingolstadt, \rule{0.3\textwidth}{0.4pt}	\\
\textcolor{white}{.}\qquad\qquad\qquad\qquad\quad \small (Datum) \\ [1.3cm]
	
(Signature) \\
Firstname Lastname

	\cleardoublepage
	
	\include{nomenclature}		
	\printnomenclature
	\cleardoublepage	
	
	
%----------Glossar/Glossary-------------------------------------------------------------
% Anzeige erst auf Tools>Glossary bei jeder Änderung!!
\newglossaryentry{abs}{name={Eric Horacsek},description={Absolute operation}}  



\newacronym{d}{D}{Dimensions}
\newacronym{rgb}{RGB}{Red Green Blue channels}




	
	\printglossaries	
	\glsaddallunused
	\cleardoublepage

	
	%Abbildungsverzeichnis/List of figures
	\thispagestyle{myPageStyle}
	\renewcommand*\listfigurename{List of figures} % Remove for German thesis
	\listoffigures
	\cleardoublepage
	
	%Tabellenverzeichnis/List of tables
	\thispagestyle{myPageStyle}
	\renewcommand*\listtablename{List of tables} % Remove for German thesis
	\listoftables
	\cleardoublepage
	
	% Inhaltsverzeichnis
	\thispagestyle{myPageStyle}
	\renewcommand{\contentsname}{Table of contents} % Remove for German thesis
	\tableofcontents
	\cleardoublepage
	\singlespacing
	
%--------------------------------------------------------------------------------	
%------Ausarbeitung--------------------------------------------------------------
%--------------------------------------------------------------------------------

	\pagenumbering{arabic} 						% Arabische Nummerierung der Kapitel/Arabic page numbering
	\include{mainpart}	

	%\shorthandon{"}
	
%--------------------------------------------------------------------------------
%-----Anhang---------------------------------------------------------------------
%--------------------------------------------------------------------------------
	
	\pagenumbering{Roman} 					% Römische Nummerierung der Kapitel/Roman page numbering
	\setcounter{page}{6} 						% Beginn bei Seitenzahl X (hier: 6) um bei oberer Nummerierung aufzuschließen/Adapt page numbering
	
	%Glossar/Glossary
	\thispagestyle{myPageStyle}
	\glssetwidest{A D A S} 						% gleicher Abstand zur 2. Spalte (längstes Wort)					
	\setglossarystyle{alttree}																	
	%\printglossary[title=Abkürzungsverzeichnis,toctitle=Abkürzungsverzeichnis] 	% Rename for German thesis
	\cleardoublepage
		
	
	%Literaturliste/Literature references
	\thispagestyle{myPageStyle}
	\bibliographystyle{plainnat}
	%\bibliographystyle{abbrv}% changed abbrvdin to abbrv % DIN-Norm für Literaturdarstellung  plaindin 
	%\bibliographystyle{abbrvnat}
	\setcitestyle{authoryear, open={(}, close={)}}
	
	\renewcommand{\refname}{Literature references} % Remove for German thesis
	\bibliography{literature}					% Pfad und Datei der Literaturdatenbank/Path and file name of literature references
	\cleardoublepage	
	
	%Anhänge/Appendices
	%\thispagestyle{myPageStyle} 
	%\include{appendices}
	%\cleardoublepage
	
%----------------------------------------------------------------------------------
%----------------DOKUMENTENENDE - END OF DOCUMENT----------------------------------
%----------------------------------------------------------------------------------
	
\end{document}
